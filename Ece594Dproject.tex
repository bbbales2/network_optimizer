\documentclass[12pt]{article}
\usepackage{fullpage}
\usepackage{amsthm}
\usepackage{amsmath}
\usepackage{amssymb}
\usepackage{listings}
\begin{document}

\title{ECE 594 D project}
\author{Scott Cesar}
\date{3-1-16}
\maketitle

\section{Pigou's network with Biases}
Consider a model of flow where units of flow have a bias for each potential route.\\
Let this bias be characterized as a Bernoulli process such that the bias for a route $i$ is either $a_i$ or $b_i$ with probability $p_i$.\\
Pigous network presents 2 routes, $r_1$ and $r_2$.\\
The cost on each edge is:
\[c_1(x) = x\]
\[c_2(x) =1\]
We can denote the effect of the biases in aggregate by $B_i(x)$ meaning the contribution of a bias given a certain amount of flow.\\
The Nash equilibrium is then 
\[\min_x \int_0^x (c_1(x) +B_1(x))dx +  \int_0^{1-x}(c_2(x) +B_2(x))dx\]
Which can be minimized by setting the derivative to 0 and solving:
\[ c_1(x) +B_1(x) +  c_2(1-x) +B_2(1-x)\]

Then note that the bias function is unknown.\\
Because the biases are Bernoulli distributions, the population falls into one of four classes:\\
\[P(a_1a_2) =p_1p_2\]
\[P(a_1b_2) =p_1(1-p_2)\]
\[P(b_1a_2) =(1-p_1)p_2\]
\[P(b_1b_2) =(1-p_1)(1-p_2)\]

Then we can separate the flow based on which section of the population it belongs to yielding A vector equation:\\
\[x= \left[ \begin{array}{c} x_1 \\x_2\\x_3\\x_4 \end{array}\right]\]
Where the bounds:
\[0 \leq x_1 \leq p_1p_2\]
\[0 \leq x_2 \leq p_1(1-p_2)\]
\[0 \leq x_3 \leq (1-p_1)p_2\]
\[0 \leq x_4 \leq (1-p_1)(1-p_2)\]
or
\[v = \left[ \begin{array}{c} p_1p_2 \\p_1(1-p_2)\\(1-p_1)p_2\\(1-p_1)(1-p_2) \end{array}\right]  \]
\[0 \leq x \leq v\]
Hold.
\[B_1(x) = [a_1,a_1,b_1,b_1] \cdot x\]
\[B_1(x) = [a_2,b_2,a_2,b_2] \cdot x\]
define
\[B_1=  [a_1,a_1,b_1,b_1]\]
\[B_2 =  [a_2,b_2,a_2,b_2]\]
Which yields a new cost function:
\[c_1(x) = [1,1,1,1] \cdot x\]
\[c_2(x) =1\]



The total traffic along each branch is bounded by:
\[\min_x \int_0^x (c_1(x) +B_1(x))dx +  \int_0^{v-x}(c_2(x) +B_2(x))dx\]
Assuming this vector valued integral is possible (Blithely) Then taking the derivative and setting it to 0 we get:
\[ c_1(x) +B_1(x) + c_2(x-v) +B_2(x-v) = 0 \]
\[  [1,1,1,1] \cdot x + B_1 \cdot x  +B_2\cdot (x-v) -1 =0\]
\[  [1,1,1,1] \cdot x + B_1 \cdot x  -B_2\cdot v +B_2 \cdot x -1 =0\]
\[  (1+ B_1 +B_2) \cdot x - b_2\cdot v -1 =0\]

\[b_2 \cdot v = a_2(p_1p_2)+ b_2p_1(1-p_2)+a_2(1-p_1)p_2 + b_2(1-p_1)(1-p_2) \]\[= a_2p_2(p_1+1-p_1) +b_2(1-p_2)(p_1+1-p_1) = a_2p_2 +b_2(1-p_2)\]

\[  (1+ B_1 +B_2) \cdot x -( a_2p_2 +b_2(1-p_2)) = 1\]
\section{}
New theories: bias can't be joined as a function of traffic? Should be possible. A person applies their specific bias, so other people do not influence the performance of the bias: the Nash equilibria is determined by my individual bias  and the current allocation of people. Then traffic is constrained by a new set of constraints:
No more traffic of a type $x_1,x_2,x_3,x_4$ can flow on branch 1 if this inequality fails:
 \[x_1: x_1+x_2+x_3+x_4 + a_1  < 1+a2 \]
 \[x_2:x_1+x_2+x_3+x_4+ a_1  < 1+b2 \]
 \[x_3: x_1+x_2+x_3+x_4 + b_1  < 1+a2 \]
 \[x_4: x_1+x_2+x_3+x_4 + b_1  < 1+b2 \]
 
That is to say if the total cost of path 1 including the traffic types bias exceeds that of path 2, no more traffic of that type can flow.\\
 \[x_1: x_1+x_2+x_3+x_4   < 1+a_2 -a_1 \]
 \[x_2:x_1+x_2+x_3+x_4  < 1+b_2 -a_1 \]
 \[x_3: x_1+x_2+x_3+x_4  < 1+a_2-b_1 \]
 \[x_4: x_1+x_2+x_3+x_4  < 1+b_2 -b_1 \]
Then these constraints are still subject to \[0 \leq x \leq v\] as well.\\
 The Nash equilibria are any sets which satisfy this constraint.\\
 The optimal Nash equilibria would be one which minimizes the global cost as well, $\min_{x\in NE} c_1(x) + c_2(1-x)$.\\ This might be mostly irrelevant? can't actually force the evolution of an optimal NE.\\
 Then in this model of Constraining the NE a tax function manifests in the form:
 \[x_1: c_1(x) +t(x)   < c_2(1-x)+t_2(1-x)+a_2 -a_1 \]
 \[x_2:c_1(x)+t(x) < c_2(1-x)+t_2(1-x)+b_2 -a_1 \]
 \[x_3: c_1(x)+t(x)  < c_2(1-x)+t_2(1-x)+a_2-b_1 \]
 \[x_4: c_1(x)+t(x)  < c_2(1-x)+t_2(1-x)+b_2 -b_1 \]
 This can be characterized as $\mathcal N(X, \vec c, \vec t, \vec b)$\\
 Where:\\
  $X$ is the matrix of edges $\times$ bias states for all the edges of the network.\\
 $\vec c$ is the vector of cost functions for each edge\\
 $\vec t$ is the vector of tax functions for each edge\\
 $\vec b$ is the vector of bias functions for each edge\\
 
 Then the goal of a tax is to minimize this feasible region over the game until this holds:
 \subsection{Actual optimum}
 The actual optimum is given by minimizing the total cost of this function: \[\min_{\vec x} \sum_{e \in G} \vec x_e c_e(\vec x_e)\] which represents the costs paid for routing $\vec x_e$ traffic across an edge $e$ times the amount of traffic routed across $e$\\ which in Pigou's network is minimizing \[\min_x x c_1(x) + (1-x)c_2(1-x)\]
 \[\min_x = x^2 + x\]
 \[\Rightarrow 2x +1 =0\]
 \[x = .5\]
 \[ \min_{\vec x} \sum_{e \in G} \vec x_e c_e(\vec x_e) = .75 \] 
 
 \subsection{Optimal Taxing}
Then we can state the definition of an optimal tax is one where
the feasible set $\mathcal N$ of the game (it's Nash Equilibria)  is equal to the optimal solution:
\[ \mathcal N(X, \vec c, \vec t, \vec b) = \min_{\vec x} \sum_{e \in G}  \vec x_e c_e(\vec x_e) \]
Then we apply this to the Bernoulli pigou's game, simplifying the $\mathcal N$ given previously to:
 \[\left[\begin{array} {c} x_1: c_1(x)    < c_2(1-x)+t_2(1-x)-t(x)+a_2 -a_1 \\ 
 x_2: c_1(x) < c_2(1-x)+t_2(1-x)-t(x)+b_2 -a_1  \\
  x_3: c_1(x)  < c_2(1-x)+t_2(1-x)-t(x)+a_2-b_1  \\
   x_4: c_1(x)  < c_2(1-x)+t_2(1-x)-t(x)+b_2 -b_1 \end{array} \right]  =  x \rightarrow .75 \]
   This becomes: the conditions:
\[\left[\begin{array} {c}
    x_1: c_1(x)    < c_2(1-x)+t_2(1-x)-t(x)+a_2 -a_1 \leq .75 \\ 
 x_2: c_1(x) < c_2(1-x)+t_2(1-x)-t(x)+b_2 -a_1 \leq .75  \\
  x_3: c_1(x)  < c_2(1-x)+t_2(1-x)-t(x)+a_2-b_1 \leq .75 \\
   x_4: c_1(x)  < c_2(1-x)+t_2(1-x)-t(x)+b_2 -b_1 \leq .75 \end{array} \right]
   \] 
 \section{Generalizing}
 Then for a general game with more than two paths and discrete biases 
 
 \section{Motivating example}
 Consider a game where 
 \[a_1 = a_2 = .25\]\[b_1 = b2 =.75\]
 \[p_1=p_2 = .5\]
 
 Which yields the constraints:
 \[v = \left[ \begin{array}{c}.25 \\ .25 \\ .25\\ .25 \end{array}\right]  \]
\[0 \leq x \leq v\]
 \[x_1: x_1+x_2+x_3+x_4   < 1 \]
 \[x_2:x_1+x_2+x_3+x_4  < 1.5 \]
 \[x_3: x_1+x_2+x_3+x_4  < .5 \]
 \[x_4: x_1+x_2+x_3+x_4  < 1 \]
 and attempts to minimize perceived cost: 
\[ \min_x \int_0^x c_1(x) dx + \int_0^(1-x) c_2(x)dx\]
\[ \Rightarrow c_1(x) - c_2(x-1) =0\] 
subject to these constraints.
As x is minimized at $x=1$ in this case, which is larger than any feasible set for the players, we can treat it as finding the maximal $x$ which is supported by the nash equilibria which is $.75$. Empirical tests match this conclusion.


\end{document}